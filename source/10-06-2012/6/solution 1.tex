\documentclass{article}
\usepackage[utf8x]{inputenc}
\usepackage[T1, T2A]{fontenc}
\usepackage[russian]{babel}
\usepackage{amsmath}
\usepackage{amssymb}
\setlength\parindent{0pt}
\usepackage[parfill]{parskip}
\pagenumbering{gobble}

\begin{document}
Будем генерировать случайную подстановку немного нестандартным образом. Сначала равновероятно выберем случайную перестановку, а затем 
сопоставим ей циклы, заканчивающиеся на элементах в порядке тривиальной перестановки. Например, для перестановок из $9$ элементов
\begin{align*}
172598346 &\to (1)(72)(5983)(4)(6)\\
592613478 &\to (59261)(3)(4)(7)(8)\\
641982735 &\to (641)(982)(73)(5)
\end{align*}
Рассмотрим самый правый элемент из элементов под номерами $1\ldots k$ в такой записи. Любой из этих элементов может с одинаковой 
вероятностью быть самым правым. Чтобы данные $k$ элементов лежали в одном цикле, самым правым из них может быть только элемент 
под номером $1$. Поэтому искомая вероятность равна $1/k$. 
\end{document}
