\documentclass{article}
\usepackage[utf8x]{inputenc}
\usepackage[T1, T2A]{fontenc}
\usepackage[russian]{babel}
\usepackage{amsmath}
\usepackage{amssymb}
\setlength\parindent{0pt}
\usepackage[parfill]{parskip}
\pagenumbering{gobble}

\begin{document}
Посчитаем, для начала, вероятность того, что данные $k$ элементов окажутся в цикле длины $k$. Всего перестановок $n!$, 
циклов $(k-1)!$, перестановок из оставшихся элементов $(n - k)!$. Тогда такая вероятность равна
$$p_k = \frac{1}{n!} (k-1)! (n-k)!$$
Пусть теперь они оказались в цикле длины $k+m$. Тогда мы дополнительно выбираем $m$ элементов из $n-k$ и вероятность становится
$$p_{k+m} = \frac{1}{n!} C^m_{n-k} (k+m-1)! (n-k-m)!$$
Полная вероятность равна сумме
$$\sum_{m=0}^{n-k} p_{k+m} = \frac{1}{n!} \sum_{m=0}^{n-k} C^m_{n-k} (k+m-1)! (n-k-m)! = $$
$$ = \frac{(n-k)!}{n!} \sum_{m=0}^{n-k} \frac{(k+m-1)!}{m!} = \frac{(n-k)!}{n!} \frac{k+m-m}{k} \sum_{m=0}^{n-k} \frac{(k+m-1)!}{m!} = $$
$$= \frac{(n-k)!}{kn!} \sum_{m=0}^{n-k} \left( \frac{(k+m)!}{m!} - \frac{m(k+m-1)!}{m!} \right) = $$
$$ = \frac{(n-k)!}{kn!} \left( \sum_{m=1}^{n-k+1} \frac{m(k+m-1)!}{m!} - \sum_{m=0}^{n-k} \frac{m(k+m-1)!}{m!} \right) = $$
$$ = \frac{(n-k)!}{kn!} \frac{(n-k+1)n!}{(n-k+1)!} = \frac{1}{k}.$$
\end{document}
