\documentclass{article}
\usepackage[utf8x]{inputenc}
\usepackage[T1, T2A]{fontenc}
\usepackage[russian]{babel}
\usepackage{amsmath}
\usepackage{amssymb}
\setlength\parindent{0pt}
\usepackage[parfill]{parskip}
\pagenumbering{gobble}

\begin{document}
Сравним массы $503$-их по массе гирек в этих группах. Пусть, для определенности, $a_{503} > b_{503}$. Тогда гирьки, которые легче $b_{503}$ в ее группе (+ сама $b_{503}$) и тяжелее $a_{503}$ в ее группе не могут быть $1006$-ми по массе, поэтому выкидываем их из групп. В результате такого шага у нас остались две группы по $503$ гирьки. Продолжая аналогичные рассуждения (в случае нечетного числа гирек в группе легкую при сравнении не выкидываем), за $10$ шагов у нас останется по одной гирьке в каждой группе и та, которая легче, и будет искомой.
\end{document}
