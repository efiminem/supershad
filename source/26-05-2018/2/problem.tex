\documentclass{article}
\usepackage[utf8x]{inputenc}
\usepackage[T1, T2A]{fontenc}
\usepackage[russian]{babel}
\usepackage{amsmath}
\usepackage{amssymb}
\setlength\parindent{0pt}
\usepackage[parfill]{parskip}
\pagenumbering{gobble}

\begin{document}
На отрезке $[0,1]$ в точках $x$, $y$, независимо выбранных из равномерного распределения, находится два детектора элементарных частиц. Детектор засекает частицу, если она пролетает на расстоянии не более $\frac{1}{3}$ от него. Известно, что поля восприятия детекторов покрывают весь отрезок. С какой вероятностью $y > \frac{5}{6}$?
\end{document}
