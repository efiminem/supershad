\documentclass{article}
\usepackage[utf8x]{inputenc}
\usepackage[T1, T2A]{fontenc}
\usepackage[russian]{babel}
\usepackage{amsmath}
\usepackage{amssymb}
\setlength\parindent{0pt}
\usepackage[parfill]{parskip}
\pagenumbering{gobble}

\begin{document}
Рассмотрим пару пользователей $a$ и $b$. Пусть они друзья. Тогда множества их друзей, не включающие $a$ и $b$, назовем $F(a)$ и $F(b)$. По условию задачи $F(a)$ и $F(b)$ не пересекаются. У любого пользователя 
из $F(a)$ существует один общий друг с $b$. Понятно, что он может быть только из $F(b)$. Таким же образом у любого пользователя из $F(b)$ существует единственный друг из $F(a)$. 
Такое может быть только в случае, если множества $F(a)$ и $F(b)$ содержат одинаковое число пользователей. Пусть теперь $a$ и $b$ не друзья. Тогда существует пользователь $c$, который 
общий для $a$ и $b$. По уже доказанному, у $a$ и $c$ одинаковое число друзей, у $b$ и $c$ одинаковое число друзей, а значит $a$ и $c$ также имеют одинаковое число друзей.
\end{document}
