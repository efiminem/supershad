\documentclass{article}
\usepackage[utf8x]{inputenc}
\usepackage[T1, T2A]{fontenc}
\usepackage[russian]{babel}
\usepackage{amsmath}
\usepackage{amssymb}
\setlength\parindent{0pt}
\usepackage[parfill]{parskip}
\pagenumbering{gobble}

\begin{document}
Подстановка $\sigma$ задана двумя массивами $\text{a[1\,.\,.\,n]}$ и $\text{b[1\,.\,.\,n]}$ состоящими из всех различных чисел от $1$ до $n$ и такими, что $\text{b[i]} = \sigma (\text{a[i]})$ для каждого $i = 1,\ldots,n$ (например, $\text{a} = \text{[2,\,\,3,\,\,1]}$, $\text{b} = \text{[1,\,\,3,\,\,2]}$ кодирует транспозицию $(1,2)$). Придумайте алгоритм, определяющий, содержит ли $\sigma$ цикл длины $k$. Ваш алгоритм может изменять исходные массивы, но должен справляться с задачей за $O(n^2)$ операций с использованием $O(1)$ дополнительной памяти (оценивая эти две асимптотики, можете считать $k$ константой).
\end{document}
