\documentclass{article}
\usepackage[utf8x]{inputenc}
\usepackage[T1, T2A]{fontenc}
\usepackage[russian]{babel}
\usepackage{amsmath}
\usepackage{amssymb}
\setlength\parindent{0pt}
\usepackage[parfill]{parskip}
\pagenumbering{gobble}

\begin{document}
Когда студент пришел в аудиторию, на доске было написано число $0$. В ожидании лекции студент
подкидывает монетку и, если выпадет орел, он прибавляет к числу $1$, а если решка --- то вычитает $1$. Орел и решка
выпадают с равной вероятностью. Найдите вероятность того, что на момент после $(2n+1)$-го подбрасывания число на доске сменило знак 
(с положительного на отрицательный или наоборот) (а) ровно $n$ раз; (б) ни разу.
\end{document}
