\documentclass{article}
\usepackage[utf8x]{inputenc}
\usepackage[T1, T2A]{fontenc}
\usepackage[russian]{babel}
\usepackage{amsmath}
\usepackage{amssymb}
\setlength\parindent{0pt}
\usepackage[parfill]{parskip}
\pagenumbering{gobble}

\begin{document}
Обозначим через $M(k)$ математическое ожидание числа бросков, требуемых для получения суммы, большей или равной $n$, при условии, что в начальный момент сумма была равна $k$.\\
Тогда имеем
$$M(n)=0, M(n-1)=1, M(n-2) = 1 + \frac{1}{n} M(n-1) = 1 + \frac{1}{n},$$
$$M(n-3) = 1 + \frac{1}{n} M(n-2) + \frac{1}{n} M(n-1) = 1 + \frac{2}{n} + \frac{1}{n^2},$$
$$M(n-4) = 1 + \frac{1}{n} M(n-3) + \frac{1}{n} M(n-2) + \frac{1}{n} M(n-1) = 1 + \frac{3}{n} + \frac{3}{n^2} + \frac{1}{n^3}.$$
Возникает гипотеза о том, что
$$M(n-k) = \sum_{i=1}^{k-1} C_{k-1}^i \frac{1}{n^i},$$
которая проверяется при помощи индукции.\\
Искомое математическое ожидание равно
$$M(0) = \sum_{i=1}^{n-1} C_{n-1}^i \frac{1}{n^i} = \left( 1 + \frac{1}{n} \right)^{n-1}.$$
\end{document}
