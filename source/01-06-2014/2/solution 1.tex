\documentclass{article}
\usepackage[utf8x]{inputenc}
\usepackage[T1, T2A]{fontenc}
\usepackage[russian]{babel}
\usepackage{amsmath}
\usepackage{amssymb}
\setlength\parindent{0pt}
\usepackage[parfill]{parskip}
\pagenumbering{gobble}

\begin{document}
Пусть $x_1$, $x_2$, $x_3$ --- три точки на окружности. С вероятностью $1$ найдется единственная пара чисел $t_2, t_3$, такая, что 
$x_1 + t_2 x_2 + t_3 x_3 = 0$. Легко убедится, что искомая вероятность совпадает с вероятностью $P(t_2, t_3 > 0)$. Рассмотрим события $A_1 = \{t_2, t_3 > 0\}$, $A_2 = \{t_2>0,t_3<0\}$, 
$A_3 = \{t_2<0, t_3>0\}$, $A_4 = \{t_2,t_3<0\}$. Так как эти события равновероятны, несовместны и их объединение имеет вероятность $1$, заключаем, что вероятность каждого из них равна $\frac{1}{4}$.
\end{document}
