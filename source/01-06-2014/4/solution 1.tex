\documentclass{article}
\usepackage[utf8x]{inputenc}
\usepackage[T1, T2A]{fontenc}
\usepackage[russian]{babel}
\usepackage{amsmath}
\usepackage{amssymb}
\setlength\parindent{0pt}
\usepackage[parfill]{parskip}
\pagenumbering{gobble}

\begin{document}
Возьмем на диаметре полукруга $n+1$ точку $A_1,A_2, \ldots A_{n+1}$ и для каждой из них зададим наш вопрос. По принципу Дирихле, для каких-то двух соседних точек полученный ответ относился к одной и той же загаданной точке. Теперь мы рассматриваем точки $B_i$ пересечения окружностей с центрами в точках $A_i$ и $A_{i+1}$, $i=1,\ldots,n$. По сказанному выше, хотя бы одна из загаданных точек совпадает с одной из точек $B_i$. Эту точку мы находим за $n$ вопросов. Итого нам потребовалось не более $(n+1)+n=2n+1$ вопросов.
\end{document}
