\documentclass{article}
\usepackage[utf8x]{inputenc}
\usepackage[T1, T2A]{fontenc}
\usepackage[russian]{babel}
\usepackage{amsmath}
\usepackage{amssymb}
\setlength\parindent{0pt}
\usepackage[parfill]{parskip}
\pagenumbering{gobble}

\begin{document}
Рассмотрим индикаторные случайные величины:
$$X_i = \begin{cases} 1,& \text{если } \sigma_i = i,\\0,& \text{иначе} \end{cases}.$$
Тогда число $X$ неподвижных точек подстановки $\sigma$ равно сумме $X_1 + X_2 + \cdots + X_n$.\\
Найдем математическое ожидание каждого из слагаемых:
$$M(X_1) = 0 \cdot P(X_1=0) + 1 \cdot P(X_1=1) = P(X_1=1)=P(\sigma_i=i)=\frac{1}{n}.$$
В самом деле, элемент $i$ фиксирует $(n-1)!$ перестановок.\\
Теперь воспользуемся линейностью матожидания:
$$M(X) = M(X_1 + X_2 + \cdots + X_n) = M(X_1) + M(X_2) + \cdots + M(X_n) = n \cdot \frac{1}{n} = 1.$$
\end{document}
