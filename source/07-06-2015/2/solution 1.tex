\documentclass{article}
\usepackage[utf8x]{inputenc}
\usepackage[T1, T2A]{fontenc}
\usepackage[russian]{babel}
\usepackage{amsmath}
\usepackage{amssymb}
\setlength\parindent{0pt}
\usepackage[parfill]{parskip}
\pagenumbering{gobble}

\begin{document}
Пусть $(\cdot,\cdot)$ --- стандартное евклидово скалярное произведение
$$(u,v) = u_1 v_1 + u_2 v_2 + \cdots + u_n v_n.$$
Заметим, что $(v \cdot v^{\mathrm{T}})x = v \cdot v^{\mathrm{T}} x = (v,x)v$. Пространство $V$, на котором действует наш линейный оператор, 
раскладывается в прямую сумму $V = < v > \oplus {< v >}^\perp$. Нетрудно видеть, что оба слагаемых являются собственными подпространствами для 
$v \cdot v^{\mathrm{T}}$ с собственными значениями $|v|^2$ и $0$ соответственно:
\begin{align*}
w = \lambda v &\Rightarrow (v \cdot v^{\mathrm{T}})w = \lambda v \cdot (v^{\mathrm{T}} v) = |v|^2 \cdot \lambda v;\\
w \perp v &\Rightarrow (v \cdot v^{\mathrm{T}})w = v \cdot (v^{\mathrm{T}} w) = (v,w)v = 0.
\end{align*}
\end{document}
