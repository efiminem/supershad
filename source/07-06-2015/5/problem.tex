\documentclass{article}
\usepackage[utf8x]{inputenc}
\usepackage[T1, T2A]{fontenc}
\usepackage[russian]{babel}
\usepackage{amsmath}
\usepackage{amssymb}
\setlength\parindent{0pt}
\usepackage[parfill]{parskip}
\pagenumbering{gobble}

\begin{document}
Электрическая цепь представляет собой связный неориентированный граф без кратных ребер,
 в котором ребра (числом $N$) --- это провода, а вершины --- либо лампочки, либо единственный источник тока. 
На каждом ребре размещено реле. Лампочка горит, если существует путь, соединяющий ее с источником тока, вдоль которого все реле находятся в положении <<включено>>.
 Известно, что ровно одно из реле бракованное и никогда не пропускает ток. Вы можете включать и отключать реле (и видите, горят ли лампочки). 
Изначально все выключатели находятся в положении <<включено>>. Опишите способ нахождения неисправного реле за $O(N)$ операций включения-выключения.
\end{document}
