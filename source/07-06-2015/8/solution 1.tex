\documentclass{article}
\usepackage[utf8x]{inputenc}
\usepackage[T1, T2A]{fontenc}
\usepackage[russian]{babel}
\usepackage{amsmath}
\usepackage{amssymb}
\setlength\parindent{0pt}
\usepackage[parfill]{parskip}
\pagenumbering{gobble}

\begin{document}
Проведем через точки $C$ и $D$ экватор. Тогда, очевидно, для выполнения условия задачи необходимо, чтобы точки $A$ и $B$ относительно этого экватора 
лежали в разных полусферах. Вероятность такого события равна $\frac12$. Пусть теперь точка $E$ -- точка пересечения этого экватора и кратчайшей дуги, 
соединяющей точки $A$ и $B$. Ясно, что ее вероятность равномерно распределена по экватору. Таким образом, задача свелась к нахождению вероятности того, 
что случайная точка $E$ на окружности (экваторе), попадет в дугу $CD$:
$$p(E \in CD) = \int\limits_0^{\pi} \frac{x}{2\pi} \frac{dx}{\pi} = \frac14.$$
Тогда полная вероятность равна:
$$p(AB \cap CD) = \frac12 p(E \in CD) = \frac18.$$
\end{document}
