\documentclass{article}
\usepackage[utf8x]{inputenc}
\usepackage[T1, T2A]{fontenc}
\usepackage[russian]{babel}
\usepackage{amsmath}
\usepackage{amssymb}
\setlength\parindent{0pt}
\usepackage[parfill]{parskip}
\pagenumbering{gobble}

\begin{document}
Рассмотрим случайную четвёрку точек $A$, $B$, $C$ и $D$. Прежде всего надо заметить, что
ситуации, когда три и более точек лежат на одной большой окружности (в частности, если
какие-то две из них диаметрально противоположны), имеют место c вероятностью ноль и
потому не заслуживают рассмотрения. Далее, рассмотрим точки $A_-$, $B_-$, $C_-$ и $D_-$,
диаметрально противоположные исходным. Имеются $16$ четвёрок точек $A_\pm$, $B_\pm$, $C_\pm$, $D_\pm$, из
которых лишь $2$ нам подходят (это легко увидеть, если предположить, что исходные
точки лежат в одной полусфере). Понятно, что появление случайных точек $A$, $B$, $C$ и $D$ можно представить, 
как появление восьми точек, включая диаметрально противоположные, а затем реализацию одной из четверок. Поскольку только $2$ 
нам подходят, вероятность равна $\frac18$. 
\end{document}
