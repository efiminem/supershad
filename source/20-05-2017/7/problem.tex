\documentclass{article}
\usepackage[utf8x]{inputenc}
\usepackage[T1, T2A]{fontenc}
\usepackage[russian]{babel}
\usepackage{amsmath}
\usepackage{amssymb}
\setlength\parindent{0pt}
\usepackage[parfill]{parskip}
\pagenumbering{gobble}

\begin{document}
Назовем матрицу \textit{вращательной}, если при повороте на $90^{\circ}$ вокруг центра она не меняется.\\
   (a) Докажите, что для любого набора чисел $\lambda_1, \ldots \lambda_k \in \mathbb{R}$ найдется $>n \in \mathbb{N}$
и вращательная матрица $n\times n$, для которой $\lambda_1, \ldots, \lambda_k$ являются собственными значениями.\\
   (б) Докажите, что у вращательной матрицы с действительными коэффициентами все собственные векторы $v$ с отличными от нуля действительными собственными значениями
симметричны (то есть $v_i = v_{n-i+1}$).
\end{document}
