\documentclass{article}
\usepackage[utf8x]{inputenc}
\usepackage[T1, T2A]{fontenc}
\usepackage[russian]{babel}
\usepackage{amsmath}
\usepackage{amssymb}
\setlength\parindent{0pt}
\usepackage[parfill]{parskip}
\pagenumbering{gobble}

\begin{document}
Так как $A$ --- симметричная и положительно определена, то она может быть представлена в виде $A=C^{\mathrm{T}}C$. Поэтому
$$
X^{\mathrm{T}}C^{\mathrm{T}}CX=(CX)^{\mathrm{T}}CX=M^{\mathrm{T}}M.\\
$$
Отсюда, учитывая что $m_{ij}=\sum\limits_{k=1}^{n}c_{ik}x_{kj}$, получаем:
$$
\mathrm{tr}(X^{\mathrm{T}}C^{\mathrm{T}}CX)=\sum\limits_{i=1}^{n}m_{ii}^2=\sum\limits_{i=1}^{n}(\sum\limits_{k=1}^{n}c_{ik}x_{ki})^2,
$$
то есть $\mathrm{tr}>0$ для любых $X$.
\end{document}
