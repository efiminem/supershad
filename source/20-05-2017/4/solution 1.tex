Заметим, что все маленькие кубики разбиваются на 4 группы:
<ol>
 <li>кубики имеющие 3 белые грани, и находящиеся в углах большого куба, их в точности 8 штук;</li>
 <li>кубики, имеющие 2 белые грани и находящиеся между кубиками 1 типа, их количество равно 12;</li>
 <li>кубики с 1 белой гранью, находящиеся в центрах граней большого куба, их 6 штук;</li>
 <li>кубик, у которого все грани черные, он находится в центре большого куба.</li>
</ol>
Теперь аккуратно посчитаем количество вариантов, при которых все грани куба будут белыми. Фиксируем верхнюю грань большого куба - $6$ способов, при каждой фиксации $4$ различных способа расположения боковых граней, таким образом для каждой конкретной комбинации кубиков, при которой они образуют белый куб, существует $24$ способа их перестановок. Для каждой группы количество равно количеству перестановок внутри группы умножить на количество поворотов кубика, и оно равно:
<ol>
 <li> $8!\cdot3^8$ - кубик можно вращать вокруг диагонали;</li>
 <li> $12!\cdot2^{12}$ - два варианта расположения белых граней; </li>
 <li> $6!\cdot4^6$ - при фиксации белой грани существует 4 варианта расположения;</li>
 <li> $1$. </li>
</ol>
Всего возможных комбинаций кубиков равно $27!\cdot27!\cdot{24}^{27}$: $27!$ - способов выбрать маленький кубик, $27!$ способов его расположения в большом и $24$ способа расположения каждого кубика при его фиксации в большом.
Таким образом требуемая вероятность равна: $$\frac{24\cdot (8!\cdot3^8) \cdot (12!\cdot2^{12}) \cdot (6!\cdot4^6) \cdot 1}{27!\cdot27!\cdot{24}^{27}}$$
