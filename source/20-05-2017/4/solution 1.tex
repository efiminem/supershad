\documentclass{article}
\usepackage[utf8x]{inputenc}
\usepackage[T1, T2A]{fontenc}
\usepackage[russian]{babel}
\usepackage{amsmath}
\usepackage{amssymb}
\setlength\parindent{0pt}
\usepackage[parfill]{parskip}
\pagenumbering{gobble}

\begin{document}
Заметим, что все маленькие кубики разбиваются на 4 группы:
\begin{enumerate}
\item кубики имеющие 3 белые грани, и находящиеся в углах большого куба, их в точности 8 штук;
\item кубики, имеющие 2 белые грани и находящиеся между кубиками 1 типа, их количество равно 12;
\item кубики с 1 белой гранью, находящиеся в центрах граней большого куба, их 6 штук;
\item кубик, у которого все грани черные, он находится в центре большого куба.
\end{enumerate}
Теперь аккуратно посчитаем количество вариантов, при которых все грани куба будут белыми. Фиксируем верхнюю грань большого куба - $6$ способов, при каждой фиксации $4$ различных способа расположения боковых граней, таким образом для каждой конкретной комбинации кубиков, при которой они образуют белый куб, существует $24$ способа их перестановок. Для каждой группы количество равно количеству перестановок внутри группы умножить на количество поворотов кубика, и оно равно:
\begin{enumerate}
\item $8!\cdot3^8$ - кубик можно вращать вокруг диагонали;
\item $12!\cdot2^{12}$ - два варианта расположения белых граней;
\item $6!\cdot4^6$ - при фиксации белой грани существует 4 варианта расположения;
\item $1$.
\end{enumerate}
Всего возможных комбинаций кубиков равно $27!\cdot27!\cdot{24}^{27}$: $27!$ - способов выбрать маленький кубик, $27!$ способов его расположения в большом и $24$ способа расположения каждого кубика при его фиксации в большом.
Таким образом, требуемая вероятность равна: $$\frac{24\cdot (8!\cdot3^8) \cdot (12!\cdot2^{12}) \cdot (6!\cdot4^6) \cdot 1}{27!\cdot27!\cdot{24}^{27}}$$
\end{document}
