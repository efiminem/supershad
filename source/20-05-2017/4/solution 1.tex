\documentclass{article}
\usepackage[utf8x]{inputenc}
\usepackage[T1, T2A]{fontenc}
\usepackage[russian]{babel}
\usepackage{amsmath}
\usepackage{amssymb}
\setlength\parindent{0pt}
\usepackage[parfill]{parskip}
\pagenumbering{gobble}

\begin{document}
Заметим, что все маленькие кубики разбиваются на $4$ группы по количеству белых граней --- $8$ кубиков с $3$ гранями, $12$ кубиков с $2$ гранями, $6$ кубиков с $1$ гранью и $1$ черный кубик.

Посчитаем число комбинаций, которые можно получить вращая маленький кубик. Существует $6$ вариантов зафиксировать верхнюю грань и $24$ варианта расположения боковых граней при каждой фиксации. Таким образом, существует $24$ способа повернуть маленький кубик. 

Для каждой группы кубиков количество подходящих вариантов равно количеству перестановок внутри группы умножить на количество подходящих поворотов:
\begin{enumerate}
\item $8!\cdot 3^8$ --- кубик можно вращать вокруг диагонали;
\item $12!\cdot 2^{12}$ --- два варианта расположения белых граней;
\item $6!\cdot 4^6$ --- при фиксации белой грани существует $4$ варианта расположения;
\item $1! \cdot 24$ --- для черного кубика подходят любые вращения.
\end{enumerate}
Всего возможных комбинаций расположений кубиков $27! \cdot{24}^{27}$.
Таким образом, искомая вероятность равна: $$\frac{8!\cdot 3^8 \cdot 12!\cdot 2^{12} \cdot 6! \cdot 4^6 \cdot 24}{27! \cdot{24}^{27}} = \frac{8!\cdot 3^8 \cdot 12!\cdot 2^{24} \cdot 6!}{27! \cdot{24}^{26}}.$$
\end{document}
