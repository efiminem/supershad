\documentclass{article}
\usepackage[utf8x]{inputenc}
\usepackage[T1, T2A]{fontenc}
\usepackage[russian]{babel}
\usepackage{amsmath}
\usepackage{amssymb}
\setlength\parindent{0pt}
\usepackage[parfill]{parskip}
\pagenumbering{gobble}

\begin{document}
Предположим, что такая непрерывная функция существует. Заметим, что в таком случае функция $f(f(x))$ строго убывает. Из этого следует, что она биективна. 
Из биективности функции $f(f(x))$ следует биективность функции $f(x)$ (сюръективность и инъективность легко обосновываются по отдельности). Непрерывная биективная 
функция должна быть строго монотонной, а из строгой монотонности функции $f(x)$ следует, что функция $f(f(x))$ строго возрастает. Получили противоречие --- значит, такой функции не существует.
\end{document}
