\documentclass{article}
\usepackage[utf8x]{inputenc}
\usepackage[T1, T2A]{fontenc}
\usepackage[russian]{babel}
\usepackage{amsmath}
\usepackage{amssymb}
\setlength\parindent{0pt}
\usepackage[parfill]{parskip}
\pagenumbering{gobble}

\begin{document}
Все города графства Орэ связаны каким-то конечным числом дорог. Если инквизитор странствует по графству достаточно долго, то он проедет достаточно много дорог, поэтому хотя бы по одной дороге AB (A и B --- города) он проедет не менее пяти раз. При этом не менее трех раз он проедет по этой дороге в одном и том же направлении (скажем, от A до B), поэтому, если из города B, кроме BA, ведут еще две дороги BC и BD, то инквизитор минимум дважды, --- скажем, после $i$-го и после $j$-го посещения города B, где $j>i$, --- сворачивал, выезжая из B (куда он оба раза приезжал из A), в одну и ту же сторону, скажем, в сторону города C. Но из условия тогда следует, что не только в $i$-e и в $j$-е посещение B инквизитор приехал в B из одного города --- из A, --- но и в A он оба раза приезжал из одного и того же города P (ведь если инквизитор после B свернул на дорогу BC, например, налево, то в A он должен был свернуть направо после посещения P). Аналогично этому устанавливается, что полностью совпадают пути инквизитора, предшествующие двум рассматриваемым посещениям города B: в город P он оба раза попал из одного и того же города, и так далее. Но тогда если инквизитор до $i$-го посещения B миновал, начиная с выезда из города Э, какое-то число $k$ городов, то и за $k$ городов до $j$-го посещения B он снова был в Э, что и доказывает утверждение задачи.
\end{document}
