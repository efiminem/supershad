\documentclass{article}
\usepackage[utf8x]{inputenc}
\usepackage[T1, T2A]{fontenc}
\usepackage[russian]{babel}
\usepackage{amsmath}
\usepackage{amssymb}
\setlength\parindent{0pt}
\usepackage[parfill]{parskip}
\pagenumbering{gobble}

\begin{document}
Перепишем выражение для членов ряда в следующем виде
$$a_n = (\ln \ln n)^{-\ln n} = \exp (-\ln n \ln \ln \ln n) = \frac{1}{n^{\ln \ln \ln n}}.$$
Заметим, что начиная с $n = \lceil \exp \exp \exp (2) \rceil$, данный ряд сходится быстрее, чем 
сходящийся ряд $\frac{1}{n^2}$. Следовательно, ряд сходится.
\end{document}
