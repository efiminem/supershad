\documentclass{article}
\usepackage[utf8x]{inputenc}
\usepackage[T1, T2A]{fontenc}
\usepackage[russian]{babel}
\usepackage{amsmath}
\usepackage{amssymb}
\setlength\parindent{0pt}
\usepackage[parfill]{parskip}
\pagenumbering{gobble}

\begin{document}
Понятно, что число линейно независимых собственных векторов, принадлежащих некоторому собственному значению (геометрическая кратность), равно алгебраической кратности этого собственного значения (поскольку есть $n$ линейно независимых собственных векторов, а геометрическая кратность не превышает алгебраическую). Тогда, при распределении $n+1$ вектора по $n$ собственным значениям, некоторому собственному значению $\lambda$ достанется на $1$ вектор больше, чем его геометрическая кратность. Пусть теперь есть собственное значение $\kappa \neq \lambda$. Тогда возьмем $n$ векторов, кроме одного вектора из $\kappa$. В таком случае эти вектора будут линейно зависимы --- противоречие. Это означает, что все собственные значения оператора равны $\lambda$.\\
Таким образом, любой вектор из собственного подпространства, при действии на него оператора, растягивается в $\lambda$ раз. Но поскольку из собственных векторов можно составить базис, любой другой вектор пространства, при действии на него оператора, также будет растягиваться в $\lambda$ раз. Нетрудно понять, что матрица такого оператора должна быть скалярной. В самом деле, при умножении матрицы на вектора вида $(0,\ldots 1 \ldots ,0)^\mathrm{T}$, мы должны получить вектора вида $(0,\ldots \lambda \ldots ,0)^\mathrm{T}$. Значит, все элементы на диагонали матрицы равны $\lambda$. Если же теперь для некоторых $i\neq j$ $a_{ij}\neq 0$, то легко подобрать вектор, который уже не растянется в $\lambda$ раз.\\
В качестве примера для любой скалярной матрицы можно взять все возможные вектора вида $(0,\ldots 1 \ldots ,0)^\mathrm{T}$ и вектор $(1,\ldots 1 \ldots, 1)^\mathrm{T}$.
\end{document}
