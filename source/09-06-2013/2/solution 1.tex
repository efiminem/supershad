\documentclass{article}
\usepackage[utf8x]{inputenc}
\usepackage[T1, T2A]{fontenc}
\usepackage[russian]{babel}
\usepackage{amsmath}
\usepackage{amssymb}
\setlength\parindent{0pt}
\usepackage[parfill]{parskip}
\pagenumbering{gobble}

\begin{document}
Рассмотрим элемент $i \in A$. Очевидно, подмножеств в $A$, содержащих $i$ и не содержащих $i$, равное количество. Таким образом, вероятность того, что $i$ лежит в $A_j$ равна $\frac12$. Эти вероятности независимы для разных $j$. Получаем, что вероятность того, что $i$ содержится в $A_1 \cap A_2 \cap \ldots \cap A_k$ равна $\frac{1}{2^k}$. Соответственно, вероятность того, что $i$ не содержится в $A_1 \cap A_2 \cap \ldots \cap A_k$ равна $1 - \frac{1}{2^k}$. Эти вероятности независимы для разных $i$. Получается, что вероятность того, что ни один из номеров $i$ не попал в $A_1 \cap A_2 \cap \ldots \cap A_k$ равна $(1 - \frac{1}{2^k})^n$.
\end{document}
