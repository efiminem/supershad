\documentclass{article}
\usepackage[utf8x]{inputenc}
\usepackage[T1, T2A]{fontenc}
\usepackage[russian]{babel}
\usepackage{amsmath}
\usepackage{amssymb}
\setlength\parindent{0pt}
\usepackage[parfill]{parskip}
\pagenumbering{gobble}

\begin{document}
Заметим, что $A$ --- симметрическая ненулевая матрица с неотрицательными элементами и нулями на диагонали. Докажем, что у такой матрицы есть отрицательное собственное значение.\\
Известный факт, что симметрическая матрица диагонализуема в вещественном базисе (все собственные значения вещественны). Допустим, что все собственные значения $A$ неотрицательны. Рассмотрим квадратичную форму $q$ с матрицей $A$ в базисе $\{e_1, \ldots, e_n\}$. Тогда эта квадратичная форма неотрицательно определена, так как все собственные значения неотрицательны. То есть $\forall v\colon q(v) \geqslant 0$. С другой стороны, пусть $a_{ij} \neq 0$. Тогда $q(e_i - e_j) = a_{ii} - 2a_{ij} + a_{jj} = -2a_{ij} < 0$. Это противоречит неотрицательной определенности $q$. Значит, исходное предположение неверно, и у $A$ есть отрицательное собственное значение. 
\end{document}
