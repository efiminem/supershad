\documentclass{article}
\usepackage[utf8x]{inputenc}
\usepackage[T1, T2A]{fontenc}
\usepackage[russian]{babel}
\usepackage{amsmath}
\usepackage{amssymb}
\setlength\parindent{0pt}
\usepackage[parfill]{parskip}
\pagenumbering{gobble}

\begin{document}
Переставляя строки, мы можем добиться того, чтобы позиции первых (слева) единиц не убывали сверху вниз. При этом определитель либо не изменится, либо поменяет знак. Если у двух строк позиции первых единиц совпадают, то вычтем ту, в которой меньше единиц из той, в которой больше. Определитель при этом не меняется. Такими операциями мы можем добиться того, что позиции первых единиц строго возрастают сверху вниз. При этом либо матрица окажется вырожденной, либо верхнетреугольной с единицами на диагонали. То есть, определитель станет либо $0$, либо $1$. Так как определитель при наших операциях либо не менялся, либо поменял знак, изначальный определитель был $\pm 1$ или $0$.
\end{document}
