\documentclass{article}
\usepackage[utf8x]{inputenc}
\usepackage[T1, T2A]{fontenc}
\usepackage[russian]{babel}
\usepackage{amsmath}
\usepackage{amssymb}
\setlength\parindent{0pt}
\usepackage[parfill]{parskip}
\pagenumbering{gobble}

\begin{document}
На станцию приходят в случайное время две электрички. Времена их приходов независимы и имеют экспоненциальное распределение с плотностью $e^{-x} \cdot \mathbb{I} \{ x > 0 \}$. Студент приходит на станцию в момент времени $2$. Найдите\\
a) вероятность того, что он сможет уехать хотя бы на одной электричке;\\
б) математическое ожидание времени ожидания студентом ближайшей электрички (считаем, что время ожидания равно нулю, если студент опоздал на обе электрички).
\end{document}
