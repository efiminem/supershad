\documentclass{article}
\usepackage[utf8x]{inputenc}
\usepackage[T1, T2A]{fontenc}
\usepackage[russian]{babel}
\usepackage{amsmath}
\usepackage{amssymb}
\setlength\parindent{0pt}
\usepackage[parfill]{parskip}
\pagenumbering{gobble}

\begin{document}
Рассмотрим функцию $F(t) = \int\limits_{a+t}^{b+t} f(x) dx$. Если $f(a) \neq f(b)$, $F(t)$ будет строго монотонной в окрестности $t=0$. Тогда существует такое $\Delta t$, что $F(\Delta t) > F(0)$. В таком случае мы можем уменьшить длину промежутка $[a+\Delta t; b + \Delta t]$ и получить $\alpha$, а значит, интервал $[a;b]$ был не минимальным.
\end{document}
