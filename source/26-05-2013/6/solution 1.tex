\documentclass{article}
\usepackage[utf8x]{inputenc}
\usepackage[T1, T2A]{fontenc}
\usepackage[russian]{babel}
\usepackage{amsmath}
\usepackage{amssymb}
\setlength\parindent{0pt}
\usepackage[parfill]{parskip}
\pagenumbering{gobble}

\begin{document}
1) Посмотрим на первую строку и первый столбец матрицы. Запомним, хорошие они или плохие. На это потребуется $O(n)$ времени и $2$ бита 
дополнительной памяти.

2) Пройдемся по оставшейся части матрицы $i=2\ldots n$, $j=2\ldots n$. Если $a_{ij} = 0$, занулим $a_{i1}$ и $a_{1j}$. Таким образом, 
в первой строке (столбце) будут стоять единицы, если вся строка (столбец) хорошие и наоборот. Этот шаг занимает $O(n^2)$ по времени, 
дополнительной памяти не требуется.

3) Пройдемся по этой же матрице еще раз. Если $a_{i1} = 0$ либо $a_{1j} = 0$, занулим $a_{ij}$. В результате этих манипуляций мы 
выполнили условия задачи для всей матрицы, кроме, возможно, первой строки (столбца). Этот шаг также занимает $O(n^2)$ по времени.

4) Используя результат пункта 1 приведем первую строку (столбец) в соответствии с условиями задачи. Это займет $O(n)$ по времени.

Таким образом, мы решили задачу за $O(n^2)$ по времени и за $O(1)$ дополнительной памяти.
\end{document}
