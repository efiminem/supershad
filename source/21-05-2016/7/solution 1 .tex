\documentclass{article}
\usepackage[utf8x]{inputenc}
\usepackage[T1, T2A]{fontenc}
\usepackage[russian]{babel}
\usepackage{amsmath}
\usepackage{amssymb}
\setlength\parindent{0pt}
\usepackage[parfill]{parskip}
\pagenumbering{gobble}

\begin{document}
Воспользуемся известным фактом, что определитель матрицы равен произведению ее собственных значений. Отсюда следует, что если мы найдем собственные значения 
матрицы $u u^{\mathrm{T}} J$ (обозначим их как $\lambda_i$), то определитель исходной матрицы будет равен $\prod\limits_i (1 + \beta \lambda_i)$.\\
Для нахождения собственных значений матрицы $u u^{\mathrm{T}} J$ воспользуемся ассоциативностью произведения матриц. Заметим, что если мы умножим матрицу на произвольный 
вектор $(u u^{\mathrm{T}} J) h = u (u^{\mathrm{T}} J h)$, то в случае $u^{\mathrm{T}} J h \neq 0$ получившийся вектор будет коллинеарен вектору $u$.\\
Отсюда следует, что\\
(а) $u$ -- собственный вектор с собственным значением $u^{\mathrm{T}} J u$,\\
(б) все собственные значения кроме, возможно, $u^{\mathrm{T}} J u$ равны $0$.\\
Функция $u^{\mathrm{T}} J u$ представляет собой квадратичную форму с нулевой матрицей, поскольку матрица $J$ кососимметрична. Значит, все собственные значения $u u^{\mathrm{T}} J$ ранвы нулю.
Таким образом, определитель исходной матрицы равен $1$.
\end{document}
