\documentclass{article}
\usepackage[utf8x]{inputenc}
\usepackage[T1, T2A]{fontenc}
\usepackage[russian]{babel}
\usepackage{amsmath}
\usepackage{amssymb}
\setlength\parindent{0pt}
\usepackage[parfill]{parskip}
\pagenumbering{gobble}

\begin{document}
Договоримся, что каждая команда за турнир получает очки, равные числу превзойденных ею команд. Сначала докажем следующую простую лемму:\\\\
\textbf{Лемма.} Пусть команда Е не превосходит команду К. Тогда К набрала больше очков, чем Е.\\\\
\textbf{Доказательство.} Если Е не превосходит К, то К победила команду Е, а также все команды, которые победила команда Е.\\
Теперь пусть Х --- команда, которую превзошла команда Е. Если Е выиграла у Х, то К также выиграла у Х. Значит, К превосходит Х. Если же Е выиграла у команды F, которая выиграла у Х, то заметим, что К тоже выиграла у F. Значит, К выиграла у F, которая выиграла у Х, то есть К превосходит Х. Итого, К превосходит все команды, которые превзошла Е, да еще и Е в придачу, то есть как минимум на одну команду больше, чем Е. Лемма доказана.\\\\
(а) Пусть А --- команда, заработавшая максимальное число очков. Докажем, что А --- чемпион. Допустим, это не так, тогда есть команда В, которую А не превзошла. По лемме получаем, что В заработала больше очков, чем А. Противоречие.\\\\
(б) Пусть у нас есть два чемпиона: А и В. Друг с другом они играли; пусть, к примеру, победила А. Так как В превосходит все другие команды (и А в частности), то В победила некоторую команду, которая выиграла у А. Допустим для начала, что есть команды, которые победили и А, и В. Тогда можно показать, что та из них (назовем ее С), которая набрала больше всего очков, и будет третьим чемпионом. В самом деле, пусть Е --- команда, которую не превзошла С. Тогда, во-первых Е победила и А, и В, а во-вторых, Е заработала больше очков, чем С --- противоречие. Пусть теперь нет команд, которые победили и А, и В. Рассмотрим множество всех таких команд, которые победили А, но проиграли В. Отметим, что оно непусто (см. выше). Среди них возьмем команду с наибольшим числом очков. Тогда пользуясь леммой мы можем установить, что эта команда является третьим чемпионом. 
\end{document}
