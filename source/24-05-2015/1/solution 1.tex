\documentclass{article}
\usepackage[utf8x]{inputenc}
\usepackage[T1, T2A]{fontenc}
\usepackage[russian]{babel}
\usepackage{amsmath}
\usepackage{amssymb}
\setlength\parindent{0pt}
\usepackage[parfill]{parskip}
\pagenumbering{gobble}

\begin{document}
Сначала докажем, что последовательность сходится. Если $a_n < 0$, то $a_{n+1} < 0$, поэтому она ограниченна сверху. Сравним $a_n$ и $a_{n+1}$:
$$a_n \;\;?\;\; \frac{a_n^2 - 3}{4} \Leftrightarrow 0 \;\;?\;\; a_n^3 - 3a_n^2 - 4a_n \Leftrightarrow 0 \;\;?\;\; a_n(a_n+1)(a_n-4).$$
Видим, что при $a_n \in (-1;0)$ имеет место неравенство $a_n < a_{n+1}$, то есть последовательность возрастает. По теореме Вейерштрасса она имеет предел. Чтобы его найти, перейдем к пределу в нашем рекуррентном соотношении:
$$a = \frac{a^2(a-3)}{4} \Leftrightarrow a^3 - 3a^2 - 4a = 0,$$
откуда предел может быть одним из чисел $0$, $-1$ и $4$. Нетрудно понять, что это $0$.
\end{document}
