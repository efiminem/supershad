\documentclass{article}
\usepackage[utf8x]{inputenc}
\usepackage[T1, T2A]{fontenc}
\usepackage[russian]{babel}
\usepackage{amsmath}
\usepackage{amssymb}
\setlength\parindent{0pt}
\usepackage[parfill]{parskip}
\pagenumbering{gobble}

\begin{document}
За столом сидят $n$ старателей, перед каждым из которых находится кучка золотого песка. Каждую минуту происходит следующее:
по общей команде каждый из них перекладывает в свою кучку половину песка из кучки левого соседа и половину --- из кучки правого соседа. Опишите асимптотическое поведение кучек (а) при $n=3$; (б) при произвольном $n$.
\end{document}
