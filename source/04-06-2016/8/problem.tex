\documentclass{article}
\usepackage[utf8x]{inputenc}
\usepackage[T1, T2A]{fontenc}
\usepackage[russian]{babel}
\usepackage{amsmath}
\usepackage{amssymb}
\setlength\parindent{0pt}
\usepackage[parfill]{parskip}
\pagenumbering{gobble}

\begin{document}
а) Докажите, что во множестве отрезков $\Lambda = \{ [i,j] | i,j=1,\ldots ,n, i<j \}$ можно
выбрать подмножество $\Sigma$, содержащее $O(n \log n)$ отрезков так, чтобы любой отрезок из $\Lambda$
представлялся в виде объединения не более двух отрезков из $\Sigma$.\\
б) Докажите, что эта оценка точна, то есть подможество $\Sigma \subseteq \Lambda$, удовлетворяющее условиям, должно содержать $\Omega (n \log n)$ отрезков.
\end{document}
