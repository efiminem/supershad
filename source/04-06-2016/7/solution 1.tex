\documentclass{article}
\usepackage[utf8x]{inputenc}
\usepackage[T1, T2A]{fontenc}
\usepackage[russian]{babel}
\usepackage{amsmath}
\usepackage{amssymb}
\setlength\parindent{0pt}
\usepackage[parfill]{parskip}
\pagenumbering{gobble}

\begin{document}
Разобьем интервал интегрирования на $n$ участков, на каждом из которых целая часть числа $nx$ постоянна. Тогда
$$\int\limits_0^1 e^{\{nx\}} x^{2016} dx = \sum_{k=1}^n \int\limits_{\frac{k-1}{n}}^{\frac{k}{n}} e^{nx} e^{1-k} x^{2016} dx.$$
Пусть функция $e^{nx} x^{2016}$ имеет первообразную $F(x)$. Тогда по формуле Ньютона-Лейбница получим
$$\sum_{k=1}^n \int\limits_{\frac{k-1}{n}}^{\frac{k}{n}} e^{nx} e^{1-k} x^{2016} dx = (e-1) \sum_{k=1}^n \frac{1}{e^k} F\left( \frac{k}{n} \right) + \frac{1}{e^n} F(1) - F(0).$$
Нетрудно понять, что $F(x)$ имеет вид $e^{nx} g(x) = e^{nx} \sum_{s=0}^{2016} A_s x^s$. Коэффициенты полинома $g(x)$ легко найти из выражения $F'(x) = e^{nx} x^{2016}$. Рекуррентно получим
$$A_{2016} = \frac{1}{n},$$
$$A_{2015} = \frac{-2016}{n^2},$$
$$A_{2014} = \frac{2015 \cdot 2016}{n^3},$$
$$\vdots$$
$$A_{0} = \frac{1 \cdots 2016}{n^{2017}}.$$
Сразу заметим, что $\lim\limits_{n\to \infty} \frac{1}{e^n} F(1) = \lim\limits_{n\to \infty} F(0) = 0$. Далее запишем
$$(e-1) \sum_{k=1}^n \frac{1}{e^k} F\left( \frac{k}{n} \right) = (e-1) \left( \frac{1}{n^{2017}} \sum_{k=1}^n k^{2016} - \frac{2016}{n^{2017}} \sum_{k=1}^n k^{2015} + \cdots + \frac{1 \cdots 2016}{n^{2017}} \sum_{k=1}^n k^0 \right).$$
Поскольку $\sum\limits_{k=1}^n k^s$ --- многочлен степени $s+1$, при $n \to \infty$ все члены суммы кроме первого стремятся к нулю. Итого
$$\lim\limits_{n \to \infty} (e-1) \left( \frac{1}{n^{2017}} \sum_{k=1}^n k^{2016} \right) = \frac{e-1}{2017}.$$
\end{document}
