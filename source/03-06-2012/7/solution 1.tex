\documentclass{article}
\usepackage[utf8x]{inputenc}
\usepackage[T1, T2A]{fontenc}
\usepackage[russian]{babel}
\usepackage{amsmath}
\usepackage{amssymb}
\setlength\parindent{0pt}
\usepackage[parfill]{parskip}
\pagenumbering{gobble}

\begin{document}
Рассмотрим все возможные элементы $x \in A_1 \cup A_2 \cup \cdots \cup A_n$. Для каждого из этих элементов построим матрицу $B_x$ следующим образом:
$$(B_x)_{ij} = \begin{cases}1,&\textrm{если }x \in A_i \cap A_j;\\0,&\textrm{иначе}.\end{cases}$$
В таком случае матрица $a$ будет являться суммой всех матриц $B_x$. Заметим, что если элемент $x$ не содержится в некотором множестве $A_k$, то $k$-ая строка и $k$-й столбец матрицы $B_x$ будут состоять из нулей. Ненулевые же строки и столбцы матрицы $B_x$ одинаковы. Отсюда следует, что определитель всех угловых миноров неотрицателен, а значит, в соответствии с критерием Сильвестра, неотрицательно определена и сама матрица $B_x$. Но тогда и матрица $a$ неотрицательно определена как сумма неотрицательно определенных матриц.
\end{document}
