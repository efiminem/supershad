\documentclass{article}
\usepackage[utf8x]{inputenc}
\usepackage[T1, T2A]{fontenc}
\usepackage[russian]{babel}
\usepackage{amsmath}
\usepackage{amssymb}
\setlength\parindent{0pt}
\usepackage[parfill]{parskip}
\pagenumbering{gobble}

\begin{document}
Пусть Улоф и Рави кинули монетку по $n$ раз. Обозначим вероятности возможных событий следующим образом:\\
1. У Рави выпало больше орлов, чем у Улофа --- $p$\\
2. У Улофа выпало больше орлов, чем у Рави --- $p$\\
3. У Рави и Улофа одинаковое количество орлов --- $1-2p$\\
Есть только два возможных случая, в которых у Рави, после того как он подросил монетку в $(n+1)$-й раз, орлов будет больше, чем у Улофа:\\
1. У Рави было больше орлов, чем у Улофа, и после $(n+1)$-го броска соотношение не поменялось --- $p$\\
2. У Рави и Улофа было одинаковое количество орлов и в $(n+1)$-й бросок выпал орел --- $\frac12 (1-2p)$\\
Поскольку два последних события несовместны, искомая вероятность равна $p + \frac12 (1-2p) = \frac12$.
\end{document}
