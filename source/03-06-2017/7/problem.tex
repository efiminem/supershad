\documentclass{article}
\usepackage[utf8x]{inputenc}
\usepackage[T1, T2A]{fontenc}
\usepackage[russian]{babel}
\usepackage{amsmath}
\usepackage{amssymb}
\setlength\parindent{0pt}
\usepackage[parfill]{parskip}
\pagenumbering{gobble}

\begin{document}
Пусть $A$ и $B$ --- две случайных булевых матрицы $n\times n$,
у которых каждый элемент равен $1$ с вероятностью $p$ (значения различных элементов не зависят друг от друга). Сколько в среднем единиц
будет в их произведении, если сложение и умножение происходят по модулю $2$?
\end{document}
